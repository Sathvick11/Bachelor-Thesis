\chapter{MOO Project}

\section{What is MOO?}
\acrfull{moo} and Simulation Driven Co - Design is an easy to use method developed for modular, automatized multi domain toolchains considering power loss, thermal calculations and 
lifetime assessments. In short, \ac{moo} is an optimization problem where we have at least two objective functions to optimize simultaneously.

This project is divided into many \textbf{action fields} wherein each action field focuses on tackling a defined problem. The table below explains more
about each of the action fields.

\newpage
\begin{center}
        \begin{table}[h!]
        \begin{tabular}{|>{\centering\arraybackslash}p{2cm}|>{\centering\arraybackslash}p{12cm}|}
        \hline
        \rowcolor{yellow!25} \Large{\textbf{Action Fields}} & \Large{\textbf{\multirow{2}{*}{\centering Objective / Purpose}}} \\
        \hline
        \centering
		Co-design, virtual modeling and credible simulation & 
        \begin{itemize}
            \item Simulation Driven Co-Design as the main method used for product development
            \item Robust design based on costs, performance and reliability
        \end{itemize}\\
        \hline
        Tools and Automation &
        \begin{itemize}
			\item Speed up concept decisions in product 
            development by providing automated simulation workflows
            \item Optimization on system level (function, costs, robustness)
            \item Roll out user-friendly workflows
            \item Robust modeling standards to achieve reproducible results
            \item Integration of available tools
        \end{itemize}\\
		\hline
        Realistic requirements and confidence of input data &
        \begin{itemize}
            \item Definition of realistic requirements(black box) reduced to what is necessary for the \ac{moo}
            \item Recording of all design-defining relevant use cases such as active driving, start-up, operation, braking etc
            \item Unify customer requirements to standardize designs 
        \end{itemize}\\
        \hline
        Cross-domain collaboration and qualification - establishing in projects & 
        \begin{itemize}
            \item Automatized, simulation-based, cost optimum product design in the of Cluster 3 Power Electronics
            \item Optimal design decisions
        \end{itemize}\\
        \hline
        \end{tabular}
        \caption{Action fields and Objectives of \ac{moo}}
        \label{action_field_table}
        \end{table}
\end{center}



\section{Modules and Workflows}

\section{What is Optislang}
Ansys Optislang is a framework for design exploration, \ac{cae} based sensitivity analysis and optimization in conjunction with any product development tool. It is a Process Integration and 
Design Optimization tool or in short, a \acrshort{pido} tool. Process Integration means to automate and orchestrate manual simulation processes and to realize
complex workflows. Design Optimization aims for better understanding of your design, optimizing the product, identify an improved design which has the desired
qualities and resulting in a best design by reliability analysis and statistical analysis.  
%Optislang \cite{optislang} is a software platform which is used for \acrshort{cae} based sensitivity analysis, multi-objective optimization, multidisciplinary optimization, evaluation of
%robustness, reliability analysis and statistical analysis.


Optislang uses several solvers to look into aspects like mechanical, technical, mathematical and any other problems. This is easier in Optislang as it provides
integration to create toolchains of many external programs like ANSYS, MATLAB, Excel, Python, CATIA and many more.
