\chapter{MOO Project}

\section{What is MOO?}
\acrfull{moo} \cite{moo} and Simulation Driven Co - Design is an easy to use method developed for modular, automatized multi domain toolchains considering power loss, thermal calculations and 
lifetime assessments. In short, \acrshort{moo} is an optimization problem where we have at least two objective functions to optimize simultaneously.

This project is divided into many \textbf{action fields} wherein each action field focuses on tackling a defined problem. The table below explains more
about each of the action fields.

\newpage
\begin{center}
        \begin{table}[h!]
        \begin{tabular}{|>{\centering\arraybackslash}p{2cm}|>{\centering\arraybackslash}p{12cm}|}
        \hline
        \rowcolor{yellow!25} \Large{\textbf{Action Fields}} & \Large{\textbf{\multirow{2}{*}{\centering Objective}}} \\
        \hline
        \centering
		Co-design, virtual modelling and credible simulation & 
        \begin{itemize}
            \item Identification of costs and relevant parameters by sensitivity analysis
            \item All relevant input parameters are determined with sufficient accuracy
            \item Necessary tool chains are available and automated
            \item Virtual models are validated
        \end{itemize}\\
        \hline
        Tools and Automation &
        \begin{itemize}
			\item Parameterized \acrshort{cad}, simplified models
			\item All necessary models are verified and can be incorporated in the workflows and provide reliable results
			\item Professional programmed code
			\item Data exchange works automated
        \end{itemize}\\
		\hline
        Realistic requirements and confidence of input data &
        \begin{itemize}
            \item Reduce the requirements given by the equipment manufacturer to what is really needed
            \item Show benefits to the customers
        \end{itemize}\\
        \hline
        Cross-domain collaboration and qualification - establishing in projects & 
        \begin{itemize}
            \item Cross-domain collaboration on the base of modelling approaches is established
            \item Competence management and qualification program is established
            \item Simulation driven co-design is established in and requested by the projects
        \end{itemize}\\
        \hline
        \end{tabular}
        \caption{Action fields and Objectives of \acrshort{moo}}
        \label{action_field_table}
        \end{table}
\end{center}

\section{What is Optislang}
Ansys Optislang \cite{optislang} is a software platform which is used for design exploration, \acrshort{cae} based sensitivity analysis and optimization in conjunction with any product development tool. It is a Process Integration and 
Design Optimization tool or in short, a \acrshort{pido} tool. Process Integration refers to automate and orchestrate manual simulation processes and to realize
complex workflows. Design Optimization aims for better understanding of your design, optimizing the product, identify an improved design which has the desired
qualities and resulting in a best design by reliability analysis and statistical analysis.  
%Optislang \cite{optislang} is a software platform which is used for \acrshort{cae} based sensitivity analysis, multi-objective optimization, multidisciplinary optimization, evaluation of
%robustness, reliability analysis and statistical analysis.


Optislang uses several solvers to look into aspects like mechanical, technical, mathematical and any other problems. This is easier in Optislang as it provides
integration to create toolchains of many external programs like ANSYS, MATLAB, Excel, Python, CATIA and many more.

\section{Modules and Workflows}
\subsection{Modules}
For better functioning of \acrshort{moo}, modules are created by the system developers. The modules are defined either in MATLAB or Python. Each module is 
focused on solving a particular problem/issue in the workflow.

\begin{table}
    \begin{tabular}{|>{\centering\arraybackslash}p{2cm}|>{\centering\arraybackslash}p{12cm}|}
        \hline
        Name of  the 
    \end{tabular}
\end{table}