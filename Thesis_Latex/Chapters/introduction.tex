\chapter{Introduction}
\section{Overview}
At our department, system developers are responsible for creating modules considering specific use cases like power loss, thermal calculation and
lifetime assessments. These modules are created in MATLAB and Python. To optimize the performance of these modules, optiSLang is used. optiSLang is a software 
tool that is used for optimization and robustness analysis. It contains algorithms to optimize the performance of the modules, verify the robustness of the 
modules and to perform sensitivity analysis. 

To add new features to the modules, the developers need to update the current module, test it locally and then push the changes to the repository.
This process is time-consuming and error-prone. To address this issue, this thesis proposes a framework which tests the module when a developer pushes the 
changes to the repository and notifies the developer if the module is not working as expected. 

\section{Objective}
The objective of this thesis is to create a framework that test the modules created by the developers. The framework should be able to test the modules which 
are created in MATLAB and Python. Since the framework needs to be automated, it should run the module in the command line and not using the \acrshort{gui}.
The framework should provide feedback to the developer regarding their latest changes to the module. Finally, the framework has to verify if the output 
files generated by the modules are present and are generated as expected.

\section{Outline}
In Chapter 2, we will discuss the basics of \acrlong{moo}, its applications in the organization and the current problems faced by the developers.

Chapter 3 will discuss the methods and best practices implemented while creating the framework. These methods will help in creating a robust framework.

Chapter 4 delves into the creation of the framework. It will discuss the introduction and importance of the framework, a brief overview of the framework,
the architecture and the execution. This chapter also explains the tools used to get the input files for the framework.

Chapter 5 will talk about the implementation of automated testing of the modules using a \acrlong{ci} pipeline.

Chapter 6 contains the results of the framework. It will discuss the results of the framework by running the tests on the modules.

Finally in Chapter 7, the thesis is concluded with a summary of the work done and the future work that can be done to improve the framework. 