\chapter{Related Work}
\section{\acrlong{ci} in Software Development}
\acrfull{ci} is a software development practice that integrates code changes frequently and tests them automatically to detect issues early in the development 
process. The concept of \acrshort{ci} was popularized by Martin Fowler in the early 2000s, emphasizing the importance of automating the build and test 
process to ensure code stability\cite{fowler2006continuous} . Jenkins\footnote{\url{https://www.jenkins.io/}} and Travis CI\footnote{\url{https://www.travis-ci.com}} 
are two of the most widely adopted CI tools that automate build processes and testing in a wide range of programming environments. These tools have laid the 
groundwork for integrating code quality checks, reducing effort of manual testing.

More recently, GitHub Actions\cite{github_actions} has gained popularity as a \acrshort{ci} tool integrated directly within GitHub repositories, enabling developers to automate 
workflows triggered by events like code pushes or pull requests. Similar to these approaches, the framework developed in this thesis leverages GitHub Actions 
to automate testing of OptiSlang modules, ensuring real-time feedback to developers\cite{8444829}.

\section{Automation in Simulation based Optimization Workflows}
Automation of simulation-based optimization workflows has a significant focus in engineering domains. Industries like automotive, aerospace, and manufacturing
rely heavily on simulation tools like MATLAB and Ansys to optimize designs. These tools are often integrated with optimization algorithms to automate the 
design process. The optimization algorithms are used to find the best design parameters that satisfy the design constraints and objectives. An article by
Bucher et al.\cite{non_linear_optimization} discusses the use of Optislang for optimization of a non-linear system. In this study, the authors use design of 
experiments to generate a response surface model and later optimize the model using a non-linear optimization algorithm. But the study does not discuss 
the testing phase nor the automation of the workflow, which this thesis aims to address by automating the creation and testing of Optislang modules.

Piero Pezze et al.\cite{sbpipe} have developed a pipeline for SBpipe\footnote{\url{https://sbpipe.readthedocs.io/en/latest/}}, a software tool for automating repetitive tasks in model development and analysis. By using
this pipeline, productivity and reliability during model development are increased. A case study by \cite{6200132} introduced an approach to automate unit testing 
in SCADA\footnote{\url{https://scada-international.com}} software. The study shows that by implementing automated testing framework, time was saved and the 
quality of the software was improved. \cite{6465286}, \cite{553698} and \cite{6319254} have also discussed the importance of automated testing framework in  
their respective studies and their advantages in improving software quality. This thesis aims to extend these ideas by proposing a Python based framework for 
automated creating and testing of Optislang modules. By doing so, the framework will address the issues like modularization, automation and scalability in 
simulation-based optimization workflows.

\section{\acrlong{ci} for Optimization tools}
Optislang is a software tool developed by Dynardo GmbH which is widely used in the industry for optimization, robustness analysis and many more. It also allows
integration with other simulation tools like MATLAB, Ansys, etc. to automate parametric studies and optimization workflows. However, few works have been done 
to automate the testing of Optislang modules.

An article by Mathworks \cite{mathworks} shows importance of using version control and automated testing for simulations. Another article by dSPACE \cite{dspace}
highlights the advantages of using \acrshort{ci} for Hardware-in-the-loop (HIL) simulations. These studies emphasize the implementation of DevOps practices 
like version control, automated testing and \acrshort{ci} in simulation-based workflows and the benefits of using these practices\cite{windriver}. 

A similar approach is proposed in this thesis to automate the creation and testing  modules in a \acrshort{ci} pipeline.

\section{Challenges and Advancements in Workflow Automation}
Automation in \acrshort{ci} pipelines, especially in simulation-based optimization workflows, poses several challenges. These include handling of input and parameters
for simulations, managing dependencies across modules and providing real-time feedback to the developers. Many of the studies discussed above have addressed 
these challenges in a general DevOps concept, yet there is a limited focus on simulation specific tools like Optislang.

The framework developed in this thesis aims to address these challenges by providing a modular approach to create and test Optislang modules. The framework 
provides a way to automate the creation of Optislang modules and test them using GitHub Actions. The framework also provides a way to verify the generated 
files by the pipeline. This feature ensures that the generated files are correct and can be used for further analysis. 