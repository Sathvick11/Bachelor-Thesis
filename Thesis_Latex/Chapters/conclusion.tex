\chapter{Summary}
\section{Conclusion}
The goal of this thesis was to develop a \acrlong{ci} framework for Optislang workflows that tests the modules and provides immediate feedback to the developer
without the user intervention. 
By implementing this framework, modules can be tested faster and reduce errors during the development process. Through the implementation of a Python-based framework, 
integrated with GitHub Actions, the workflow has been successfully automated, simplifying the development process for system developers at Bosch

Throughout the project, key efforts were directed toward automating the execution and testing of these modules, which were previously handled manually. 
By incorporating DevOps best practices into the framework, the process has become much more efficient, offering immediate feedback whenever changes are 
pushed to the repository. This not only saves time but also helps minimize human error during development.

The work presented here is particularly relevant as industries increasingly depend on automation to stay competitive. With the growing complexity of modern 
software and optimization tools like OptiSlang, automating repetitive tasks like testing ensures that companies can move faster and deliver more reliable 
products, which is crucial for companies like Bosch.

Ultimately, the takeaway from this thesis is clear: integrating CI practices into simulation-based environments can greatly enhance productivity, reliability, 
and scalability. This work demonstrates how automation can play a vital role in ensuring the success of complex workflows, enabling developers to focus more 
on innovation and less on routine tasks.

\section{Future Work}
The framework developed in this thesis is a starting point for automating Optislang workflows. Future work could focus on expanding the framework to support 
more modules and tools. For example, the framework could be extended to support for workflows that involve multiple modules, or to integrate with other 
optimization tools. Additionally, the framework could be enhanced to support more advanced testing. This changes can be implemented easily by extending the 
functionality of the framework. 

Some of the older modules in the framework do not consist the data for verifying output files. This is a crucial step in the framework to verify the output 
files generated by the module. By standardizing the files in every module, the framework can be used for all the modules in the repository.

Storage of input files in this thesis is done through OpenShift and Fast\acrshort{api} for a proof of concept. This can be improved by using a more secure and 
scalable solution. One possible solution is to store the files in a Blob storage in Azure\footnote{\url{https://azure.microsoft.com/en-us/products/storage/blobs/}}, 
securing the files and making it easier to access the files.